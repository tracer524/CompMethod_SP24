\documentclass[UTF8]{ctexart}
\usepackage[left=2cm,right=2cm,top=2cm]{geometry}
\usepackage{amsmath}
\usepackage{enumitem}
\usepackage{float}
\usepackage{threeparttable}
\usepackage{caption}
\usepackage{multirow}
\usepackage{graphicx}
\usepackage{listings}
\usepackage{color}
\definecolor{dkgreen}{rgb}{0,0.6,0}
\definecolor{gray}{rgb}{0.5,0.5,0.5}
\definecolor{mauve}{rgb}{0.58,0,0.82}
\lstset{frame=tb,
  language=Python,
  aboveskip=3mm,
  belowskip=3mm,
  showstringspaces=false,
  columns=flexible,
  basicstyle={\small\ttfamily},
  numbers=left,%设置行号位置none不显示行号
  %numberstyle=\tiny\courier, %设置行号大小
  numberstyle=\tiny\color{gray},
  keywordstyle=\color{blue},
  commentstyle=\color{dkgreen},
  stringstyle=\color{mauve},
  breaklines=true,
  breakatwhitespace=true,
  escapeinside=`,%逃逸字符(1左面的键),用于显示中文例如在代码中`中文...`
  tabsize=4,
  extendedchars=false %解决代码跨页时,章节标题,页眉等汉字不显示的问题
}

\setlength\lineskiplimit{5.25bp}
\setlength\lineskip{5.25bp}

\title{计算方法第三次编程作业报告}
\author{崔士强 PB22151743}
\date{\today}

\bibliographystyle{plain}

\begin{document}

\maketitle
\section{问题描述}
给定两个矩阵,利用反幂法求出它们的按模最小特征值和特征向量.

\section{问题分析}
本程序利用反幂法求解,也即利用幂法求原矩阵逆矩阵的特征向量. 
另外为了防止出现数值问题,采取规范运算,即在每个迭代中对$X^{(k)}$的每个元素除以$X^{(k)}$的最大分量. 
计算公式如下所示
\begin{equation}
  \begin{cases}
    Y^{(k)} = X^{(k)}/||X^{(k)}||_\infty \\
    AX^{(k+1)} = Y^{(k)}
  \end{cases}
  k = 0, 1, \cdots
\end{equation}

其中在求解$X^{(k+1)}$时对$A$使用Doolittle分解,化成$L \cdot U$的形式再求解.

\section{实验结果}
两个矩阵的计算结果分别如下表所示
\begin{table}[H]
  \centering
  \begin{tabular}{c|ccccc}
    \hline\hline
    $k$ & \multicolumn{5}{c}{$Y^{(k)}$} \\
    \hline
    0 & 1 & 1 & 1 & 1 & 1 \\
    1 & 0.5625 & $-1$ & 0.5625 & $-0.107143$ & 0.00446429 \\
    2 & 0.49103 & $-1$ & 0.658639 & $-0.151467$ & 0.00798141 \\
    3 & 0.490426 & $-1$ & 0.65975 & $-0.152097$ & 0.00804664 \\
    \hline\hline 
  \end{tabular}
\end{table}
\begin{table}[H]
  \centering
  \begin{tabular}{c|ccccc|c}
    \hline\hline
    $k$ & \multicolumn{5}{c|}{$X^{(k+1)}$} & $\lambda$\\
    \hline
    0 & 630 & $-1120$ & 630 & $-120$ & 5 & 0.000892857 \\
    1 & 146253 & $-297849$ & 196175 & $-45114.4$ & 2377.25 & $3.35741\times 10^{-6}$ \\
    2 & 149113 & $-304047$ & 200595 & $-46244.7$ & 2446.56 & $3.28896\times 10^{-6}$ \\
    3 & 149157 & $-304142$ & 200661 & $-46261.1$ & 2447.54 & $3.28794\times 10^{-6}$ \\
    \hline\hline 
  \end{tabular}
\end{table}

对于$A_1$,经过4轮迭代后得到按模最小特征值为$3.28794\times10^{-6}$.

对应特征向量为$(0.490426, -1, 0.65975, -0.152097, 0.00804664)^T$

\begin{table}[H]
  \centering
  \begin{tabular}{c|cccc}
    \hline\hline
    $k$ & \multicolumn{4}{c}{$Y^{(k)}$} \\
    \hline
    0 & 1 & 1 & 1 & 1 \\
    1 & 0 & 1 & 0 & 0.5 \\
    2 & $-0.111111$ & 1 & $-0.422222$ & 0.622222 \\
    3 & $-0.115543$ & 1 & $-0.425034$ & 0.624484 \\
    4 & $-0.115725$ & 1 & $-0.425687$ & 0.624769 \\
    5 & $-0.115732$ & 1 & $-0.425694$ & 0.624774 \\
    6 & $-0.115732$ & 1 & $-0.425695$ & 0.624775 \\
    \hline\hline 
  \end{tabular}
\end{table}
\begin{table}[H]
  \centering
  \begin{tabular}{c|cccc|c}
    \hline\hline
    $k$ & \multicolumn{4}{c|}{$X^{(k+1)}$} & $\lambda$\\
    \hline
    0 & 0 & 2 & 0 & 1 & 0.5 \\
    1 & $-0.625$ & 5.625 & $-2.375$ & 3.5 & 0.177778 \\
    2 & $-0.933333$ & 8.07778 & $-3.43333$ & 5.04444 & 0.123796 \\
    3 & $-0.93621$ & 8.08992 & $-3.44378$ & 5.05433 & 0.123611 \\
    4 & $-0.936712$ & 8.09382 & $-3.44549$ & 5.05681 & 0.123551 \\
    5 & $-0.936719$ & 8.09386 & $-3.44551$ & 5.05684 & 0.123551 \\
    6 & $-0.93672$ & 8.09386 & $-3.44552$ & 5.05684 & 0.12355 \\
    \hline\hline 
  \end{tabular}
\end{table}

对于$A_2$,经过7轮迭代后得到按模最小特征值为$0.12355$.

对应特征向量为$(-0.115732, 1, -0.425695, 0.624775)^T$
\section{结果分析}
\subsection{收敛速度}
对于两个矩阵,迭代次数分别为4和7,另外从按模最小特征值来看,$A_1$远小于$A_2$,因此在这个例子中有“按模最小特征值越接近于0,收敛越快”.
\subsection{数值问题}
实验过程中并未出现数值问题. 另外可以注意到,在对$A_1$进行处理时,作规范化处理时如果除以向量的无穷范数,分母数值很小,从而增大误差. 而这个无穷范数实际上是上一轮迭代得到的特征值的倒数,因此可以利用$Y^{(k)} = X^{(k)}\lambda^{(k-1)}$进行迭代.
\bibliography{math}

\end{document}
\iffalse
\begin{figure}[h]
    \centering
    \includegraphics[scale=0.5]{name.png}
    \caption{name}
\end{figure}
\fi

\lstset{
    language=Python,
    basicstyle=\small\ttfamily,
    keywordstyle=\color{blue},
    commentstyle=\color{green},
    stringstyle=\color{red},
    showstringspaces=false,
    breaklines=true,
}